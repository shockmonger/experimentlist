% Created 2014-08-01 Fri 17:04
\documentclass[11pt]{article}
\usepackage[utf8]{inputenc}
\usepackage[T1]{fontenc}
\usepackage{fixltx2e}
\usepackage{graphicx}
\usepackage{longtable}
\usepackage{float}
\usepackage{wrapfig}
\usepackage{soul}
\usepackage{textcomp}
\usepackage{marvosym}
\usepackage{wasysym}
\usepackage{latexsym}
\usepackage{amssymb}
\usepackage{hyperref}
\tolerance=1000
\providecommand{\alert}[1]{\textbf{#1}}

\title{VLabs for Music Experiment List}
\author{Tejaswinee Kelkar}
\date{\today}
\hypersetup{
  pdfkeywords={},
  pdfsubject={},
  pdfcreator={Emacs Org-mode version 7.9.3f}}

\begin{document}

\maketitle

\setcounter{tocdepth}{3}
\tableofcontents
\vspace*{1cm}

\section{Experiments}
\label{sec-1}
\subsection{Expt 1 Pitch discrimination}
\label{sec-1-1}
\subsubsection{Description:}
\label{sec-1-1-1}

This experiment will focus on students being able to distinguish
between two pitches that are played consecutively. The students will
only to answer whether the two pitches played are the same or
different.
\subsubsection{Skills:}
\label{sec-1-1-2}

Before being able to train the ear for identifying pitches, this
simple experiment will help students distinguish between pitches that
are far apart and the ones that are spaced nearer to each other.
\subsubsection{Levels:}
\label{sec-1-1-3}
\subsection{Expt 2 Pitch Direction}
\label{sec-1-2}
\subsubsection{Description:}
\label{sec-1-2-1}

This subsequent experiment focuses on being able to identify the
direction of difference of the given pitches. After the student is
comfortable telling two tones apart, this experiment will help them
understand whether the subsequent tone is lower or higher than the
first.
\subsubsection{Levels:}
\label{sec-1-2-2}
\subsection{Expt 2.5 What are notes?}
\label{sec-1-3}
\subsubsection{Description:}
\label{sec-1-3-1}

This experiment will explain the concept of notes and
octaves. Students will learn to identify octaves and figure out how
notes played in a row sound. Different instruments can be used to
elaborate this. For vocalists, this may include clicking on the names
of svaras / intervals and then listening to them. Students will also
be able to play notes as scales with note names to familiarize with
the concept of naming notes.
\subsubsection{Skills:}
\label{sec-1-3-2}

Before we move to interval identification, this experiment is a free
exploration of musical hearing.
\subsubsection{Levels:}
\label{sec-1-3-3}
\begin{itemize}

\item 1. Free exploration with instrument and note names
\label{sec-1-3-3-1}%

\item 2. Free exploration hearing different scales and note sequences\\
\label{sec-1-3-3-2}%
\end{itemize} % ends low level
\subsection{Expt 3 Pitch identification from same tonic}
\label{sec-1-4}
\subsubsection{Description:}
\label{sec-1-4-1}

Here we start to move to interval hearing and tone training. 
\subsubsection{Skills:}
\label{sec-1-4-2}

Before being able to train the ear for identifying pitches, this
simple experiment will help students distinguish between 
\subsubsection{Levels:}
\label{sec-1-4-3}


\begin{enumerate}
\item Levels
\begin{enumerate}
\item Level 1: Sa, Ma, Pa
\item Level 2: + Ga, Ni
\item Level 3: + Re, Dha
\item Level 4: All Shuddha Notes
\item Level 5: + Re Dha Komal
\item Level 6: + Ga Ni Komal
\item Level 7: + Tivra Ma
\item Level 8: All notes
\end{enumerate}
\item Number of Questions
\end{enumerate}
\subsection{Expt 4 Pitch identification from separate tonic}
\label{sec-1-5}
\subsubsection{Description:}
\label{sec-1-5-1}

While the tanpura is playing in another key, a different reference
note will be given to judge another pitch interval from.
\subsubsection{Skills:}
\label{sec-1-5-2}

This experiment is to build hearing independence outside the tonic, as may be required in some forms of light classical singing.
\subsection{Expt 5 Identifying a chain of pitches}
\label{sec-1-6}
\subsubsection{Description:}
\label{sec-1-6-1}

In this experiment, the experimenter will have to name the notes in a row of pitches in a single tonic.
\subsubsection{Skills:}
\label{sec-1-6-2}

This experiment will decelop the knowledge of `svar-sthan' in the users.
\subsubsection{Levels:}
\label{sec-1-6-3}


\begin{enumerate}
\item Levels
\item Number of Questions per level
\end{enumerate}
     
\subsection{Expt 6 Tabla Simulation}
\label{sec-1-7}
\subsubsection{Description:}
\label{sec-1-7-1}

The goal of this experiment is to simulate a tabla skin on a computer,
and choose and handshape, and simulate the sounds that a tabla makes
if it is struck at different points.
\subsubsection{Skills:}
\label{sec-1-7-2}

Understanding tabla bols also requires an understanding of how it is
played on the skin membrane. This experiment will help people who
don't have access to tablas to explore the instrument in detail.
\subsubsection{Levels:}
\label{sec-1-7-3}


\begin{enumerate}
\item Hearing bol and location of playing
\item Composing new taals
\end{enumerate}
     
\subsection{Expt 7 Khali and Taali}
\label{sec-1-8}
\subsubsection{Description:}
\label{sec-1-8-1}

This experiment is to explain the polarity between these two events in
Taal. The rhythmic and motional feeling of a downbeat and an upbeat
will be explained. Participants will get to freely explore and listen
to tabla sounds, and figure out the presence of khali and tali in the taals.
\subsubsection{Skills:}
\label{sec-1-8-2}
\subsubsection{Levels:}
\label{sec-1-8-3}


\begin{enumerate}
\item Analyzing khali and taali in visual form
\item Building larger metrical structures for khali and tali
\item Splitting These structures into different laya combinations - aad, kuaad, bayaad
\end{enumerate}
\subsection{Expt 8 Phrase hearing}
\label{sec-1-9}
\subsubsection{Description:}
\label{sec-1-9-1}

In this experiment, students will be able to distinguish between
different phrases. There are 4 kinds of basic phrases in HCM. The
participants will be expected to categorize each phrase in any of
these four types, and in the next level compose phrases belonging to
these types.
\subsubsection{Levels:}
\label{sec-1-9-2}


\begin{enumerate}
\item Differentiating between different types of phrases
\item Creating phrases having different envelopes
\end{enumerate}
\subsection{Expt 9 Intonation}
\label{sec-1-10}
\subsubsection{Description:}
\label{sec-1-10-1}


\begin{enumerate}
\item Hearing different kinds of intonation
\end{enumerate}
   

      
\subsection{Expt 10 Tuning a Tanpura}
\label{sec-1-11}
\subsubsection{Description:}
\label{sec-1-11-1}

\begin{enumerate}
\item Two tanpuras, one fixed, cranking the other one up / down
\end{enumerate}
\subsubsection{Levels:}
\label{sec-1-11-2}

   Tuning to the same tone
   Tuning to an Octave
   Tuning to a Fifth
     
\subsection{Expt 11 Singing and melograph plotting}
\label{sec-1-12}
\subsubsection{Description:}
\label{sec-1-12-1}

Dynamic capture of musical voice and generation of melographs dynamically with 
\subsubsection{Skills:}
\label{sec-1-12-2}

Melographs are a visual representation of the sung music. This
experiment will help people familiarize themselves with a visual
scheme to understand music. This scheme will be taken forwards
\subsubsection{Levels:}
\label{sec-1-12-3}


\begin{enumerate}
\item Matching own melograph with template melographs
\item Difference measure for melographs
\end{enumerate}
\section{Appendix}
\label{sec-2}
\subsection{Export drones for 6 seconds in all 12 keys}
\label{sec-2-1}
\subsection{Instruments:}
\label{sec-2-2}

\begin{enumerate}
\item Flute
\item Sitar
\item Sarod
\item Santoor
\item Sarangi
\item Violin
\item Shehnai
\item Veena
\item Voice-Male
\item Voice-Female
\end{enumerate}

\end{document}
